# siswslayer
Demo of _spatial interpolation soft weight sharing layer_. 
It is a compromise between the locally connected layers as in ["DeepFace"](https://research.fb.com/wp-content/uploads/2016/11/deepface-closing-the-gap-to-human-level-performance-in-face-verification.pdf) 
and the _hard_ weight sharing that is usually found in 
convolutional layers.

## Explanation
Instead of the global weight sharing that forces all weights to be exactly the same irrespective of the
spatial location, this layer uses kernels that are each given a 'centroid'. For brevity, we will refer to these units as
kernel centroid pairs (KCP). What we want is that 
the activations of neurons near a KCP's centroid are close to those that would have been obtained when using
the KCP's kernel, and less like the activations that would have been obtained when using the kernels of more distant 
KCPs. Additionally, when a spatial location is somewhere in between two centroids, it should inherit some of
the weights of both kernel centroid pairs.

We can do accomplish this by defining a similarity function that is defined on the spatial domain of the convolutional
output. The column and row indices will be translated to cartesian coordinates which will be used to compute local 
weighting coefficients. For now this similarity function is some exponential similarity function.

The kind of soft weight sharing is accomplished by using multiple KCPs. Each KCP first computes 
its activation, just like a single convolutional layer would do. Then, we linearly combine the 
results of these KCPs by using spatial 
coefficients which are determined by the spatial cells in the output tensors with respect 
to the centroids of each KCP. As a result, when a spatial cell is close to a 
certain KCP's centroid, its `local' kernel will look most like the kernel of that 
particular KCP. Note that each locally determined output is still factored by 
several convolution kernels. Hence, to a certain extent, one could regard this as soft weight 
sharing or distance based weight sharing. This should result in a gradual change of local 
kernels when moving from centroid to centroid, allowing the features to be significantly 
more complex with relatively few extra parameters. Put mathematically, we define the output 
of a spatial weight sharing layer as follows:
\begin{equation}
\mathsf{Y} = f\left( \sum_{s=0}^{S-1} \bigg(\mathsf W_s  \ast \mathsf X \oplus \boldsymbol b_s\bigg) \odot \mathsf S_s \right),
\end{equation}
where <img src="http://www.sciweavers.org/tex2img.php?eq=%5Cmathsf%20Y&bc=White&fc=Black&im=jpg&fs=12&ff=arev&edit=0" align="center" border="0" alt="\mathsf Y" width="15" height="15" /> is the output tensor, 
<img src="http://www.sciweavers.org/tex2img.php?eq=%5C%7B%5Cmathsf%20W_s%5C%7D_s&bc=White&fc=Black&im=jpg&fs=12&ff=arev&edit=0" align="center" border="0" alt="\{\mathsf W_s\}_s" width="57" height="18" />  are rank 4 weight 
tensors, 
<img src="http://www.sciweavers.org/tex2img.php?eq=%24%5Cmathsf%20X%24&bc=White&fc=Black&im=jpg&fs=12&ff=arev&edit=0" align="center" border="0" alt="$\mathsf X$" width="14" height="14" />
is the rank 4 input tensor,
<img src="http://www.sciweavers.org/tex2img.php?eq=%24%5Cmathsf%20X%24&bc=White&fc=Black&im=jpg&fs=12&ff=arev&edit=0" align="center" border="0" alt="$\mathsf X$" width="14" height="14" />
is a rank 4 similarity tensor 
with <img src="http://www.sciweavers.org/tex2img.php?eq=%5Cmathsf%20S_%7Bb%2Ci%2Cj%2Cc%7D%20%3D%20%5Cmathcal%20S%28i%2Cj%2Ci_s%2Cj_s%29&bc=White&fc=Black&im=jpg&fs=12&ff=arev&edit=0" align="center" border="0" alt="\mathsf S_{b,i,j,c} = \mathcal S(i,j,i_s,j_s)" width="157" height="21" />
where <img src="http://www.sciweavers.org/tex2img.php?eq=%24%28i_s%2C%20j_s%29%24&bc=White&fc=Black&im=jpg&fs=12&ff=arev&edit=0" align="center" border="0" alt="$(i_s, j_s)$" width="49" height="19" />
parameterizes the 
centroid of the s-th shared convolution. Note that 
 <img src="http://www.sciweavers.org/tex2img.php?eq=%24%5Coplus%24%20&bc=White&fc=Black&im=jpg&fs=12&ff=arev&edit=0" align="center" border="0" alt="$\oplus$ " width="17" height="14" />
 and<img src="http://www.sciweavers.org/tex2img.php?eq=%24%5Codot%24&bc=White&fc=Black&im=jpg&fs=12&ff=arev&edit=0" align="center" border="0" alt="$\odot$" width="17" height="14" />
 denote 
element-wise operations with optional broadcasting along singleton dimensions. The 
function <img src="http://www.sciweavers.org/tex2img.php?eq=%24%5Cmathcal%20S%24&bc=White&fc=Black&im=jpg&fs=12&ff=arev&edit=0" align="center" border="0" alt="$\mathcal S$" width="14" height="14" />
gives the `similarity' of a spatial cell located at 
<img src="http://www.sciweavers.org/tex2img.php?eq=%20%24%28%5Ccdot%2Ci%2Cj%2C%5Ccdot%29%24&bc=White&fc=Black&im=jpg&fs=12&ff=arev&edit=0" align="center" border="0" alt=" $(\cdot,i,j,\cdot)$" width="61" height="19" />

with respect to some convolution's centroid 
<img src="http://www.sciweavers.org/tex2img.php?eq=%24%28i_s%2C%20j_s%29%24&bc=White&fc=Black&im=jpg&fs=12&ff=arev&edit=0" align="center" border="0" alt="$(i_s, j_s)$" width="49" height="19" />. 
It is interesting to see whether the network can also learn optimal centroids by also varying <img src="http://www.sciweavers.org/tex2img.php?eq=%24%28i_s%2C%20j_s%29%24&bc=White&fc=Black&im=jpg&fs=12&ff=arev&edit=0" align="center" border="0" alt="$(i_s, j_s)$" width="49" height="19" />.

## Content of this repo
This repo contains a minimal example demonstration in which the layer is used for the MNIST digit classification task. 
This task is perhaps less suitable than the frontalized faces data that Taigman et al. worked on, because the local 
specialization of neurons by means of these KCPs will be more beneficial if the spatial structure across the images in
the data is roughly consistent, which is not the case for MNIST. Nevertheless, the visualizations in TensorBoard
help to understand what's going on!

## Running the example
To run the code with default parameters, simply do:
```bash
python3 train.py
```

Optionally, you might want to check out the other parameters:
```
usage: train.py [-h] [--color_coding]
                [--centroid_grid CENTROID_GRID [CENTROID_GRID ...]]
                [--n_centroids N_CENTROIDS] [--logdir LOGDIR]
                [--centroids_trainable CENTROIDS_TRAINABLE]
                [--log_verbosity LOG_VERBOSITY]
                [--n_filters N_FILTERS [N_FILTERS ...]]

Demonstration of the soft spatial weight sharing layer

optional arguments:
  -h, --help            show this help message and exit
  --color_coding        Whether to use color coding in TensorBoard
                        visualizations
  --centroid_grid CENTROID_GRID [CENTROID_GRID ...]
                        Grid in which the centroids are arranged at
                        initialization
  --n_centroids N_CENTROIDS
                        If n_centroids is given, the centroids are initialized
                        randomly
  --logdir LOGDIR       Specify dir for TensorFlow logs, defaults to [project_folder]/tensorboard
  --centroids_trainable CENTROIDS_TRAINABLE
                        If given, the centroid positions will be trainable
                        parameters
  --log_verbosity LOG_VERBOSITY
  --n_filters N_FILTERS [N_FILTERS ...]
```

So you could do something like:
```bash
python3 train.py --centroids_trainable  --n_filters 24 48 --centroid_grid 3 3
```

This runs the program in which the centroids are included as trainable parameters, the first convolutional layer uses
24 output features, the second convolutional layer uses 48 output features and the centroid grid is initialized as 3 x 3.

## TensorBoard
TODO